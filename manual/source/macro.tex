\usepackage{booktabs}
\usepackage[utf8]{inputenc}
\newcommand{\sectionbreak}{\clearpage}
\usepackage{amsmath,amsfonts,amsthm,mathrsfs,xspace,graphicx}
\usepackage{tocloft}
\usepackage{tikz}
\usepackage{setspace}

% Bibliography styling
\usepackage{etoolbox} 
\usepackage{url}

% Packages for generating filler text
\usepackage{lipsum}
\usepackage{blindtext} 
\usepackage{mwe}

\usepackage[linktocpage, colorlinks, bookmarks=true, citecolor=blue, urlcolor=black]{hyperref}
\usepackage{endnotes}
\usepackage{color}
\usepackage{float}
\usepackage{xcolor}
\definecolor{since}{rgb}{0.5,0.5,0.5}
\definecolor{newred}{HTML}{ED2024}
\definecolor{newgreen}{HTML}{109A48}
\definecolor{newblue}{HTML}{535DAA}
\definecolor{neworange}{HTML}{F79420}
\usepackage{mdframed}
\usepackage{bbm}
\usepackage{suffix} \usepackage{times}
\usepackage{tabularx}
\usepackage{makecell}
\usepackage{amssymb,latexsym}
\usepackage[capitalize]{cleveref}
\usepackage[font=footnotesize,labelfont=bf]{caption}
\usepackage{enumitem}
\usepackage{tikz}
\usepackage{tikz-cd}
\usepackage{multirow}

\usepackage[section]{placeins}
\usepackage[affil-it]{authblk}

\makeatletter
\renewcommand*\env@matrix[1][*\c@MaxMatrixCols c]{%
  \hskip -\arraycolsep
  \let\@ifnextchar\new@ifnextchar
  \array{#1}}
\makeatother

\usepackage[margin=1in]{geometry}

\usepackage{endnotes}
\usepackage{amssymb,latexsym}
\usepackage{enumitem}
\setlist{itemsep=0mm}
\usepackage{tikz}
\usepackage{float}
\usepackage{setspace}
\usepackage{soul}

\usepackage{thmtools}
\usepackage{thm-restate}

\usepackage{mathtools, physics}
\usepackage{complexity}

\newclass{\QPCP}{QPCP}
\newclass{\QCPCP}{QCPCP}
\newclass{\QCMAcomp}{QCMA-complete}
\newclass{\sharpP}{\#P}

\newcommand\numberthis{\addtocounter{equation}{1}\tag{\theequation}}

\usepackage[linesnumbered,ruled,vlined]{algorithm2e}
\usepackage{verbatim}

\DeclarePairedDelimiter\floor{\lfloor}{\rfloor}

\usepackage{amsthm}
\newtheorem{theorem}{Theorem}%[section] %uncomment to include section numbers in the numbering
\newtheorem*{theorem*}{Theorem}
\newtheorem{proposition}[theorem]{Proposition}
\newtheorem*{proposition*}{Proposition}
\newtheorem{fact}[theorem]{Fact}
\newtheorem*{fact*}{Fact}
\newtheorem{lemma}[theorem]{Lemma}
\newtheorem*{lemma*}{Lemma}
\newtheorem{claim}[theorem]{Claim}
\newtheorem{corollary}[theorem]{Corollary}
\newtheorem{conjecture}{Conjecture}
\newtheorem*{conjecture*}{Conjecture}

\theoremstyle{definition}
\newtheorem{definition}[theorem]{Definition}
\newtheorem*{definition*}{Definition}
\newtheorem{problem}{Problem}
\newtheorem{question}{Question}
\newtheorem{protocol}{Protocol}
\newtheorem{observation}[theorem]{Observation}

\theoremstyle{remark}
\newtheorem{remark}[theorem]{Remark}
\newtheorem*{remark*}{Remark}
\newtheorem*{example}{Example}
\newtheorem*{note}{Note}


\newcommand{\CC}{\ensuremath{\mathbb{C}}}
\newcommand{\NN}{\ensuremath{\mathbb{N}}}
\newcommand{\FF}{\ensuremath{\mathbb{F}}}
\newcommand{\KK}{\ensuremath{\mathbb{K}}}
\newcommand{\RR}{\ensuremath{\mathbb{R}}}
\newcommand{\ZZ}{\ensuremath{\mathbb{Z}}}

\newcommand{\mcA}{\ensuremath{\mathcal{A}}}
\newcommand{\mcB}{\ensuremath{\mathcal{B}}}
\newcommand{\mcC}{\ensuremath{\mathcal{C}}}
\newcommand{\mcG}{\ensuremath{\mathcal{G}}}
\newcommand{\mcO}{\ensuremath{\mathcal{O}}}
\newcommand{\mcP}{\ensuremath{\mathcal{P}}}
\newcommand{\mcS}{\ensuremath{\mathcal{S}}}
\newcommand{\mcT}{\ensuremath{\mathcal{T}}}

\DeclareMathOperator\im{im}
\DeclareMathOperator\CSS{CSS}
\DeclareMathOperator\N{N}
\DeclareMathOperator\stab{Stab}
\DeclareMathOperator\unitary{U}
\DeclareMathOperator\evenodd{par}
\DeclareMathOperator\Span{Span}
\newcommand{\ham}{\mathcal{H}}

\newcommand{\destab}{\ensuremath{e^{-i\frac{\pi}{8}Y}}}
\newcommand{\destabn}{\ensuremath{(\destab)\n}}
\newcommand{\destabdagger}{\ensuremath{e^{i\frac{\pi}{8}Y}}}
\newcommand{\destabdaggern}{\ensuremath{(\destabdagger)\n}}

\newcommand{\Hn}{\ensuremath{\ham^{(n)}}}
\newcommand{\dHn}{\ensuremath{\Tilde{\ham}^{(n)}}}
\newcommand{\Hzero}{\ensuremath{\ham_0^{(n)}}}
\newcommand{\dHzero}{\ensuremath{\Tilde{\ham}_0^{(n)}}}
\newcommand{\ketzero}{\ensuremath{\ket{0}^{\otimes n}}}
\newcommand{\brazero}{\ensuremath{\bra{0}^{\otimes n}}}
\newcommand{\n}{\ensuremath{^{\otimes n}}}

\newcommand{\local}{\ensuremath{C}}

\newcommand{\NLXS}{\text{NL$X$S} }

% \newcommand{\wt}{\ensuremath{\text{wt}}}
\DeclareMathOperator\wt{wt}

% \newcommand{\had}{\ensuremath{\text{\normalfont H}}}
\DeclareMathOperator\had{H}
% \newcommand{\eye}{\ensuremath{\mathbb{I}}}
\DeclareMathOperator\eye{\mathbb{I}}
% \newcommand{\phase}{\ensuremath{\text{P}}}
\DeclareMathOperator\phase{P}
% \newcommand{\T}{\ensuremath{\text{T}}}
\DeclareMathOperator\T{T}
% \newcommand{\CNOT}{\ensuremath{\text{CNOT}}}
\DeclareMathOperator\CNOT{CNOT}


\usepackage{accents}
\newcommand{\uhat}{\underaccent{\check}}


\newcommand{\tnorm}[1]{\norm{#1}_1}
\newcommand{\highlightname}[3]{%
  {\textcolor{#3}{\footnotesize{\bf (#1:} {#2}{\bf ) }}}}


\newcommand\restr[2]{{% we make the whole thing an ordinary symbol
  \left.\kern-\nulldelimiterspace % automatically resize the bar with \right
  #1 % the function
  % pretend it's a little taller at normal size
  \right|_{#2} % this is the delimiter
  }}
  
  